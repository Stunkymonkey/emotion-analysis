\documentclass[compress,english]{beamer}
\usepackage{emotionanalysis}

% Please add your name here
\author{Firstname Lastname}

\begin{document}

% add the topic here that you work on
% choose from:
% Face
% Images
% Body Postures
% Robotics
% Speech
% Physiological Sensors
% Generation
% Related concepts
% Application
\section{Topic}
\begin{frame}
  \frametitle{Short name of the paper}
  \begin{itemize}
  \item {\footnotesize \blueemph{Author Name 1, Author Name 2, \ldots (year): Paper title, Proceedings/Journal name. \url{https://www.url.com}}}
  \item \blueemph{Motivation}: One sentence about the research question/motivation of this paper
  \item \blueemph{Data}: How did they get data, or which data did they use.
  \item \blueemph{Method}: A short description of the method and how they applied it.
  \item \blueemph{Main Result}: Mention the main result.
  \end{itemize}  
\end{frame}


\section{Example Slide}

\begin{frame}
  \frametitle{Stance Sentiment Corpus}
  \begin{itemize}
  \item {\footnotesize \blueemph{Schuff, Barnes, et al. (2018): Annotation, modelling and analysis of fine-grained emotions on a stance and sentiment detection corpus, WASSA. \url{https://www.aclweb.org/anthology/W17-5203/}}}
  \item \blueemph{Motivation}: Generate a multi-label Twitter corpus
    with emotions. Understand which model architecture perform
    well. Understand interactions of emotions with sentiment and
    stance labels.
  \item \blueemph{Data}: Reannotation of existing stance-Twitter
    corpus from SemEval 2018, approximately 4870 Tweets.
  \item \blueemph{Method}: Corpus analysis and Application of Bag-of-words Maxent, SVM, LSTM, BiLSTM, CNN
  \item \blueemph{Main Result}: Embeddings-based methods show higher
    recall than models with discrete features. Sentiment polarity
    labels correlated with positivity and negativity of emotions, but
    there are interesting outliers.
  \end{itemize}
\end{frame}

\end{document}
